\chapter{State of the Art}
\label{cha:relatedwork}
\begin{itemize}
	\item Digital Twin General
	\begin{itemize}
		\item SHORT! Histroy of Digital Twin
		\item Digital Twins are described as (Singh et al. 2021):
		\begin{itemize}
			\item Digital Model
			\item Layout
			\item Doppelganger
			\item Clone
			\item Software analog
			\item Simulation
		\end{itemize}
		\item Application areas
		\begin{itemize}
			\item Vehicle
			\item Aircraft
			\item Machine
			\item Product
			\item Device
			\item Process
		\end{itemize}
		\item Features of a Digital Twin (Singh et al. 2021):
		\begin{itemize}
			\item In depth analysis of the real twin
			\item Design and Validation of the real twin or a new real twin (Saratha et al. 2021)
			\item Simulation of the health condition of a real twin (Tracking the Status of the Physical twin throughout its lifetime) -> predictive maintenance
			\item -> resutls Increase of savety and reliability
			\item Realtime Control of the real twin (Zitierung Fraglich) 
		\end{itemize}
	\end{itemize}
	\item Controlling with a Digital Twin and its implications
	\begin{itemize}
		\item Real Time Communication Protocolls
	\end{itemize}
\end{itemize}

There are two approaches to the creation of a digital twin. One is the manual approach to create a Architecture based on the physical Model (Ashtari Talkhestani et al. 2019; Jiang et al. 2022; Redelinghuys et al. 2020; Schroeder et al. 2021; Tekinerdogan und Verdouw 2020). This results in very specialized architecture and a lengthy Development process of the digital twin additionally to the already lengthy Modelling process within a Project Development Lifecycle. But there is the opportunity to create a digital twin from the Systems Models (Schroeder et al. 2016; Shangguan et al. 2019). This topic currently follows the same approach of having a model and creating a digital twin for that Model.

\begin{itemize}
	\item Models are Computer parse able and thus an automatic generating of a digital twin should be possible(Dalecke et al. 2022)
\end{itemize}