\chapter{Introduction}
\label{cha:introduction}
Within modern development it is aimed for the development process to be as efficient as possible. Hereby the topic of digital twin is a highly discussed topic. Digital twins are an representation of an object in reality (Brandtstaedter et al. 2018; Batty 2018). The object in reality is then called real twin. A digital twin contains the objects structure and can also contains the functional blocks for a simulation (Brandtstaedter et al. 2018; Batty 2018). It is also to consider, that a digital twin can be everything from a component inside a computer to a complete production plant.

The goal of this work is to have a digital twin automatically generated from a system modelling language. For this work the modelling language SysMD is chosen. SysMD is a systems modelling language (Dalecke et al. 2022). For this work SysMD is used, because it is an by the chair cyberphysical systems (CPS) at the Rhineland-Palatinate Technical University Kaiserslautern (RPTU) developed and presented, thus the tooling around the language exists and is with this application extended (Dalecke et al. 2022). Since the language is developed by the chair this allows us also to extend the language.

Until now the term system is now often called, but we need to define the term system in the context of this work. A system S is consists of a set of things T and relations R between the things (Klir 2001). This results in the following mathematical description (Klir 2001, S. 5):
\begin{equation}
S=(T,R)
\end{equation}
This definition allows the system representation as graph (Klir 2001), which will be important later (chapter Approach). Since this system definition is not completely compatible with the SysMD definition, the term things will be called component C thus the Mathematical definition of the system is (Klir 2001):
\begin{equation}
S=(C,R)
\end{equation}
This allows us to achieve a mathematical modelling of the structure from the real twin. This definition allows also to achieve a modelling of every system, which can also be a computer, another component or even a production plant (Brandtstaedter et al. 2018; Batty 2018). Defining a component, it is to consider, that a component inside a System can also described like a system, thus for a component it also holds the Systems formula (Klir 2001), but if the component is so basic, that this component is modeled with a basic mathematically formular $f_C (x_1…x_n )$, which results in the following mathematical representation.
\begin{equation}
	C=(C,R)
\end{equation}
To be a component, a component needs to be part of the system otherwise, it will automatically be one system.

As already stated, this work will publish an approach to an automatically generated digital twin structure, that can then be used for system analysis, functional modelling and speeding up the development process of new systems. This work is structured into the following chapters. Chapter 1 is the introduction, that aims to introduce the reader into the topic. The chapter 2 explains the state of the art, followed by the approach, which will take a deep dive into the modelling of this approach to the digital twin. In the chapter of the implementation, the challenges and approaches of the implementation of the digital twin application are displayed. After the implementation chapter, the results are discussed, and a conclusion is formulated. Finally, this work is concluded by the future work.
